% \begin{center}
\begin{table}[h]
% \begin{center}
\centering
\begin{tabular}{l|l l l c}
{Level} & {Type}        & {Produced by} &  {Storage } & {Format} \\ \hline
% \bf 0  & Sampled antenna        & Antenna & None & Binary \\
%        & voltages               & sub-arrays & & \\
\bf 1a & Ring buffer data                & $1^{\rm st}$ stage beam & 4 months$^*$ & UDP stream/ \\
       &                                 & former &  & HDF5  \\
\bf 1b & Beam-formed data                & $2^{\rm nd}$ stage beam & 4 months$^*$ & HDF5 \\
       &                                 & former & &  \\
\bf 2  & Time integrated                 & All sites & Archived & \HDF \\
       & correlated data & & & \\
\bf 3a & Physical parameters             & All sites & Archived & \HDF \\
\bf 3b & 3D-voxel parameters             & Operations centre & Archived & \HDF \\
\bf 4  & Derived geophysical & Users & Users & Publications etc \\
       & parameters & & & 
\end{tabular}
\caption{Summary of the \ED data levels. 
The \ED \DCs will receive, serve and archive all data at levels 2 and 3.
{Data used in research should be given Persistent Identifiers (PIDs) according to a common standard such as DOI, DataCite, or similar, to be unambiguously citable in publications.}
(*)~A 4~months period is selected as this is the estimated time required
to perform a ``real-time'' analysis on low-level data.
A portion of the 
level~1 data will also be archived permanently, on the order of $1\%$
of the level~1 data rate, e.g. one beam per site and/or bandwidth-limited
data.
\label{tab:datalevels}}
% \end{center}
\end{table}
% \end{center}
